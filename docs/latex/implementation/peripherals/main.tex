\subsubsection{Peripherals}
Each peripheral has its own way of communicating with our system.
Each one also runs in a separate thread for performance reasons.

We initialize them in the \verb|main.cpp| file.
\begin{center}
    \includegraphics[width=\textwidth]{images/code/main.png}
\end{center}

Notice how the creation of the \verb|CAN| thread has a different syntax.
This is because the \verb|Can| class is a functor. This was done to showcase
the different syntax possible in \verb|C++|.

To achieve this behavior the \verb|CAN| class is defined in the following way:
\begin{center}
    \includegraphics[width=\textwidth]{images/code/can_class.png}
\end{center}

The other peripherals are represented by plain functions, since it's a simpler solution.

\paragraph{CAN}
\verb|Control Area Network (CAN)| is a protocol commonly used both in automotive and industrial applications.
It is knows for being simple to wire and resistant to electrical interference.

The \verb|CAN| module shows up on \verb|Linux| as a network interface.
To communicate with it the \verb|SocketCAN API| can be used, which is an official part of
the \verb|Linux kernel|. It simplifies \verb|CAN| interactions, by abstracting it away to being
a network socket.~\cite{socket-can-manual}

To initialize this peripheral, the socket needs to be bound in the following way:
\begin{center}
    \includegraphics[width=\textwidth]{images/code/can_bind.png}
\end{center}

Configuring the socket is done using system calls like verb|ioctl| and \verb|fcntl|.
\verb|ioctl| allows us to interact with hardware devices.
In this case we use it to fetch the details of a network device called ``can0``.
\verb|fnctl| allows us to manipulate read and write locks on files.
In this case it's used to set the socket to non-blocking mode.~\cite{advanced-linux-programming}

After binding the socket is ready to be used. In this project it is used to read driver inputs and lap times
from the steering wheel controller. The telemetry also sends information like speed and voltages
back to the steering wheel to be displayed to the driver.
% TODO show picture of steering wheel

These actions are performed in a \verb|while(true)| loop. That loop is responsible for two distinct tasks.
The first task is reading the information being sent by the steering wheel controller.
\begin{center}
    \includegraphics[width=\textwidth]{images/code/can_run_read.png}
\end{center}

The second task is writing data to the steering wheel controller.
This data is then displayed to the driver on a built-in display.
\begin{center}
    \includegraphics[width=\textwidth]{images/code/can_run_write.png}
\end{center}
\paragraph{CSV}
The data readings get stored in \verb|/home/root/logs/data_n.csv|, where \verb|n|
gets incremented on every reboot.

This is done in order to not overwrite the data collected during last run,
as well as to separate the reboots from each other.

The code responsible for creating these files is the following:
\begin{lstlisting}
std::string logDir = std::string("/home/root/logs");
std::filesystem::create_directories(logDir);

// Find the next available numbered file: data_1.csv, data_2.csv, etc.
int fileIndex = 1;
std::string filename;
do {
    filename = logDir + "/data_" + std::to_string(fileIndex) + ".csv";
    fileIndex++;
} while (std::filesystem::exists(filename));

std::ofstream output(filename, std::ios::app);

if (!output.is_open()) {
    std::cout << "CSV: Unable to open csv file - " << filename << std::endl;
    return nullptr;
}
\end{lstlisting}

After this the \verb|CSV| header is written to the newly created file.

\begin{lstlisting}
output << std::fixed << std::setprecision(10);
output << "time,"
        << "timeBeforeTransmit,"
        << "accessoryBatteryVoltage,"
        << "accessoryBatteryCurrent,"
        ...
        << "motorControllerEnableOutput"
        << std::endl;
\end{lstlisting}

And finally the file is written to every 10ms with the latest data.
\begin{lstlisting}
while (true) {
    auto now = std::chrono::system_clock::now();
    auto timestamp = std::chrono::system_clock::to_time_t(now);

    output << static_cast<int>(timestamp) << ","
            << get_timeBeforeTransmit() << ","
            << get_accessoryBatteryVoltage() << ","
            << get_accessoryBatteryCurrent() << ","
            ...
            << (get_motorControllerEnableOutput() ? "true" : "false")
            << std::endl;

    std::this_thread::sleep_for(std::chrono::milliseconds(10));
}
\end{lstlisting}
% TODO rs485
\paragraph{RS485}
A serial protocol commonly used in industrial applications

% TODO gps
% TODO lora
% TODO other peripherals