\paragraph{CAN}
\verb|Control Area Network (CAN)| is a protocol commonly used both in automotive and industrial applications.
It is knows for being simple to wire and resistant to electrical interference.

The \verb|CAN| module shows up on \verb|Linux| as a network interface.
To communicate with it the \verb|SocketCAN API| can be used, which is an official part of
the \verb|Linux kernel|. It simplifies \verb|CAN| interactions, by abstracting it away to being
a network socket.~\cite{socket-can-manual}

To initialize this peripheral, the socket needs to be bound in the following way:
\begin{center}
    \includegraphics[width=\textwidth]{images/code/can_bind.png}
\end{center}

Configuring the socket is done using system calls like verb|ioctl| and \verb|fcntl|.
\verb|ioctl| allows us to interact with hardware devices.
In this case we use it to fetch the details of a network device called ``can0``.
\verb|fnctl| allows us to manipulate read and write locks on files.
In this case it's used to set the socket to non-blocking mode.~\cite{advanced-linux-programming}

After binding the socket is ready to be used. In this project it is used to read driver inputs and lap times
from the steering wheel controller. The telemetry also sends information like speed and voltages
back to the steering wheel to be displayed to the driver.
% TODO show picture of steering wheel

These actions are performed in a \verb|while(true)| loop. That loop is responsible for two distinct tasks.
The first task is reading the information being sent by the steering wheel controller.
\begin{center}
    \includegraphics[width=\textwidth]{images/code/can_run_read.png}
\end{center}

The second task is writing data to the steering wheel controller.
This data is then displayed to the driver on a built-in display.
\begin{center}
    \includegraphics[width=\textwidth]{images/code/can_run_write.png}
\end{center}