\subsection{Base System}
The base system was built using a custom \verb|Yocto| image. It required several custom recipes to work.
These recipes and the changes they introduce are documented in the following sections.

\subsubsection{Peripherals}
The system uses several peripherals that require additional hardware configuration.

Instead of using a conventional \verb|BIOS| the \verb|RaspberryPI| uses a file
called \verb|/boot/firmware/config.txt|. This file is read by the \verb|GPU| before the \verb|CPU|
initializes. This file can be used to configure hardware options, \verb|devicetree| overlays and other features.~\cite{raspberry-pi-manual}

For this system the following changes to \verb|rpi-config_git.bbappend| were required for the added peripherals to work:
\begin{center}
    \includegraphics[width=\textwidth]{images/yocto/rpi_conf.png}
\end{center}

An issue that arises with the \verb|LoRa| module is that it is treated as a \verb|tty| device.
This means that it by default receives kernel logs and consequently sends them via \verb|LoRa|.
This behavior is undesired, so it required disabling the kernel logs via \verb|rpi-cmdline.bbappend| recipe.
\begin{center}
    \includegraphics[width=0.6\textwidth]{images/yocto/rpi_cmdline.png}
\end{center}

This fix is suboptimal, since it disables logs for all \verb|tty|, but no better solution was found in the development process.

\subsubsection{Networking}

The system has a static address ethernet port, as well as a \verb|CAN| network interface.\\

On a modern system this configuration is usually done either using \verb|systemd-networkd| or \verb|NetworkManager|.
These tools are very powerful and user-friendly, but unnecessarily large for embedded applications.\\

This system uses a light-weight legacy method. It's configured using \verb|ifup/ifdown| via \verb|/etc/network/interfaces| file.
\begin{center}
    \includegraphics[width=\textwidth]{images/yocto/interfaces.png}
\end{center}

For debugging purposes the system allows connecting over \verb|SSH|.
It uses \verb|Dropbear|, a minimal \verb|SSH Server|.\\

The interface has a static address of \verb|192.168.1.100 /24|.
To make the interface hotpluggable \verb|ifplugd| was used with the following config file.
\begin{center}
    \includegraphics[width=0.6\textwidth]{images/yocto/ifplugd.png}
\end{center}

In order to use this custom configuration file it was necessary to append to the original recipe with a \verb|ifplugd_%.bbappend| file.
\begin{center}
    \includegraphics[width=\textwidth]{images/yocto/ifplugd_recipe.png}
\end{center}

For \verb|Dropbear| to work it is required to add a password to the system.
This was done by adding the following section to the main \verb|hydrogreen-image.bb| file:
\begin{center}
    \includegraphics[width=\textwidth]{images/yocto/set_password.png}
\end{center}

\subsubsection{Graphical interface}
During the development of the project a possibility of using cameras as rear view mirrors of the vehicle was explored.

% TODO x11
% TODO get pictures of display and camera used

The camera image would be displayed on a \verb|Waveshare 7" LCD| touchscreen.
To support that display the following section needs to be added to the \verb|config.txt| file.
\begin{center}
    \includegraphics[width=0.8\textwidth]{images/yocto/rpi_disp.png}
\end{center}

% TODO check command for showing camera image on screen (does it work?)
To show the camera output on screen we can run \verb|mpv /dev/video0|.

\subsubsection{Libraries}
Yocto offers built-in recipes for some of the libraries used by the telemetry application.
One of such libraries is \verb|libgpiod|, a library used to interact with \verb|GPIOs| on Linux systems.
To include this library we simply added it as a dependency of our project.

For libraries that don't have any existing recipes it is required to write a custom one.
In the case of the telemetry application, \verb|capnproto| was needed.
To add it to the project a \verb|capnbroto_1.1.0.bb| recipe file was created with the following contents:
\begin{center}
    \includegraphics[width=\textwidth]{images/yocto/capnproto.png}
\end{center}

This recipe clones the official \verb|capnproto| github repository and builds it from source using \verb|cmake|.
