\subsection{Main board}
There are multiple options for the main board.
The following section will serve as a short overview of each option
and provide reasoning behind the final choice.

% TODO show pictures of Beaglebone, RaspberryPI and the final board with modules attached

\paragraph{Custom Single Board Computer}
Developing a custom linux based system is the best solution for a large scale project.
It gives the developer team full control over the system hardware. It also allows for much smaller designs by integrating the used peripherals into the same board.
It also maximizes the profits for mass manufacturing.

Designing such boards requires significant knowledge in the domain of PCB design.
To run well linux needs to be run on an application microprocessor, such as a \textbf{Cortex-A}.

Boards for such chips are usually more complex than microcontroller boards.
Microprocessors often require several voltage levels that need to be supplied from external regulators, while microcontrollers almost universally have a single level fixed supply voltage.
Another source of complexity is that microprocessors usually need external RAM.~\cite{jay-carlson-embedded-linux}

In the case of small projects such as this one the benefits of mass production
don't apply, so it would actually be more expensive.

The difficulty involved in developing such a system could prove too substantial
for small teams due to lack of modules as well as the increased complexity
involved in custom hardware.

The added complexity makes this a suboptimal solution for this project.

\paragraph{Beaglebone Black}
This board is a popular option for both hobbyist and industry specialists.

While it is a very powerful option its downsides are that its expensive and
lacks a hobbyist oriented community.

\begin{figure}[!htb]
    \centering
    \includegraphics[width=0.6\textwidth]{images/hardware/beaglebone.png}\par
    \caption{Beaglebone Black}
    \label{fig:beaglebone}
\end{figure}

\paragraph{EdgeBox-RPI}
This board is based on the \verb|RasperryPI| chip. It has been designed for industrial applications.
It offers built-in \verb|RS485|, \verb|RS232| and \verb|CAN| support. It also features an \verb|RTC|.

It would be a great choice for an industrial settings, but in this case the price was too high and
the community around it was too small.
%TODO cite website

\begin{figure}[!htb]
    \centering
    \includegraphics[width=0.6\textwidth]{images/hardware/edgebox.png}\par
    \caption{EdgeBox-RPI}
    \label{fig:edgebox}
\end{figure}

\paragraph{RaspberryPI}
This series of boards features a wide range of possible hardware. This provides the user with a wide range of possible system peripherals.
This project uses a RaspberryPI 4b.

% TODO cite manufacturer
TODO energy comparison

It has a great community of both hobbyists and professionals around it, which
makes development much simpler.

It also features many already built hardware modules which makes it possible to
extend the capability of the board with additional peripherals. Such modules
will be used throughout this project.

\begin{figure}[!htb]
    \centering
    \includegraphics[width=0.6\textwidth]{images/hardware/rp4.jpg}\par
    \caption{RaspberryPI 4}
    \label{fig:rp4}
\end{figure}