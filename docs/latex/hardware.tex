\clearpage
\section{Hardware}

TODO give literature for these Technologies

\large\textbf{Main board}\\
There are multiple options for the main board. The following section will serve as a short overview of each option and provide reasoning behind the final choice.
\begin{itemize}
    \item[]
          \large\textbf{Custom Single Board Computer}\\
          Developing a custom linux based system is the best solution for a large scale project.
          It gives the developer team full control over the system hardware. It also allows for much smaller designs by integrating the used peripherals into the same board.
          It also maximizes the profits for mass manufacturing.

          In the case of small projects such as this one the benefits of mass production
          don't apply, so it would actually be more expensive.

          The difficulty involved in developing such a system could prove too substantial
          for small teams due to lack of modules as well as the increased complexity
          involved in custom hardware.

          That is why it is a suboptimal solution for this project.

    \item[]
          \large\textbf{Beaglebone Black}\\
          This board is a popular option for both hobbyist and industry specialists.

          While it is a very powerful option its downsides are that its expensive and
          lacks a hobbyist oriented community.

    \item[]
          \large\textbf{RaspberryPI}\\
          This series of boards features a wide range of possible hardware. This provides the user with a wide range of possible system peripherals.
          This project uses a RaspberryPI 4b.

          TODO energy comparison

          It has a great community of both hobbyists and professionals around it, which
          makes development much simpler.

          It also features many already built hardware modules which makes it possible to
          extend the capability of the board with additional peripherals. Such modules
          will be used throughout this project.
\end{itemize}

\large\textbf{Modules}\\
As mentioned this project takes advantage of multiple ready-made RaspberryPI modules available on the market.

\begin{itemize}
    \item[]
          \large\textbf{sb-components LoRa HAT}\\
          This module provides a simple way to transmit data over LoRa. This is how information is sent to the base station.
    \item[]
          \large\textbf{Waveshare RS485 CAN HAT (B)}\\
          This module provides both CAN and RS485 interfaces, which are necessary for communicating with the electrical components of the vehicle.
    \item[]
          \large\textbf{GPS GU-502GGB}\\
          This module receives NMEA packets via GPS. Such packets contain geographic coordinates, speed and even the current time.
          This data is crucial for analyzing a race. It lets the racing team optimize fuel usage throughout the track.
\end{itemize}