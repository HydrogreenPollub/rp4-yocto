\clearpage
\section{Introduction}

With rising concerns about the ecology and sustainability of fossil fuels, the
automotive industry is increasingly turning to alternative sources of power.
Manufacturers have already made multiple variants of commercially viable
electric vehicles (EVs). These vehicles however have some major flaws that
could potentially be remedied by hydrogen fuel cell technologies.

Electric vehicles often have much lower driving ranges than their standard fuel
counterparts. Storing energy in the form of hydrogen is much lighter then in
batteries~\cite{hydrogen-fuel-cell-technology}.

Another advantage of hydrogen vehicles over electric ones is the fueling time.
EVs often take up to half an hour to charge. Hydrogen vehicles only take a
couple of minutes. % TODO cite something

The goal of this project is to create a telemetry system for a hydrogen vehicle
that participates in the shell eco marathon competition. With the nature of the
competition it is crucial to focus on optimizing the vehicle as much as
possible.

Changing the different parameters of the vehicle blindly would potentially
yield some results. It would however be very time consuming without any
additional information. In order to analyze the the performance of the vehicle,
a telemetry system was needed.

The information gathered will help track the performance of the hydrogen fuel
cell, as well as the different voltages in real time. This fact is crucial not
only for research and development purposes, but it is also incredibly helpful
during the race. It lets the racing team judge the situation with great
accuracy and decide when it is possible to push the vehicle to its limits and
when caution should be taken.
