\clearpage
\section{Linux}

Initially using a linux based system for a race telemetry application might seem excessive. It however provides several benefits that make it superior to bare-metal.

Using linux provides a filesystem as well as a user environment that makes it easy to develop and debug the application.

Another large advantage of using linux instead of bare-metal is the built-in support for a large selection of hardware.
It means that development efforts can be focused on the high level logic instead of low level protocol details.~\cite{jay-carlson-embedded-linux}

There are various linux distributions that are well suited for a RaspberryPI
embedded system.

%TODO write a bit about each system
\begin{itemize}
    \item[]
          \large\textbf{Raspbian}\\
    \item[]
          \large\textbf{Ubuntu Code}\\
    \item[]
          \large\textbf{Debian}\\
    \item[]
          \large\textbf{OpenWrt}\\
    \item[]
          \large\textbf{DietPI}\\
\end{itemize}

In order to have better control over the hardware this project uses a custom
distribution. The most common ways of building such distributions are:
\begin{itemize}
    \item[]
          \large\textbf{From scratch}\\
          Building a custom linux distribution from scratch is a possibility. It gives the developer complete control over his system.
          It is also a great way to gain insight on the inner workings of Linux.~\cite{linux-from-scratch}

          However for this project that control will not be required. Instead it would
          add unnecessary complexity.
    \item[]
          \large\textbf{Buildroot}\\
          This option is one of the most popular on the market. It is often viewed as the simplest one.
          What makes it not suitable for this project is the structure of the config file being inherently unordered.
          Another factor is the documentation being very difficult to read both due to its style and presentation.
    \item[]
          \large\textbf{Yocto}\\
          While this option is definitely more complex than buildroot it features much better documentation, as well as a thriving community
          that makes development much easier.
\end{itemize}