\clearpage
\section{Linux}

There are various distros that are well suited for a RaspberryPI embedded
system.

TODO write a bit about each system
\begin{itemize}
    \item[]
          \large\textbf{Raspbian}\\
    \item[]
          \large\textbf{Ubuntu Code}\\
    \item[]
          \large\textbf{Debian}\\
    \item[]
          \large\textbf{OpenWrt}\\
    \item[]
          \large\textbf{DietPI}\\
\end{itemize}

In order to have better control over the hardware this project uses a custom
distribution. The most popular ways of building such distributions are:
\begin{itemize}
    \item[]
          \large\textbf{From scratch}\\
          Building a custom linux distribution from scratch is a possibility. It gives the developer complete control over his system.
          However for this project that control will not be required. Instead it would add unnecessary complexity.
    \item[]
          \large\textbf{Buildroot}\\
          This option is one of the most popular on the market. It is often viewed as the simplest one.
          What makes it not suitable for this project is the structure of the config file being inherently unordered.
          Another factor is the documentation being very difficult to read both due to its style and presentation.
    \item[]
          \large\textbf{Yocto}\\
          While this option is definitely more complex than buildroot it features much better documentation, as well as a thriving community
          that makes development much easier.
\end{itemize}